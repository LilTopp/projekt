% Metódy inžinierskej práce

\documentclass[10pt,twoside,slovak,a4paper]{article}

\usepackage[slovak]{babel}
%\usepackage[T1]{fontenc}
\usepackage[IL2]{fontenc} % lepšia sadzba písmena Ľ než v T1
\usepackage[utf8]{inputenc}
\usepackage{graphicx}
\usepackage{url} % príkaz \url na formátovanie URL
\usepackage{hyperref} % odkazy v texte budú aktívne (pri niektorých triedach dokumentov spôsobuje posun textu)

\usepackage{cite}
%\usepackage{times}

\pagestyle{headings}

\title{Využitie Blockchainu v hernom svete
\thanks{Semestrálny projekt v predmete Metódy inžinierskej práce, ak. rok 2022/23, vedenie: Igor Stupavský}}

\author{Daniel Činčura\\[2pt]
	{\small Slovenská technická univerzita v Bratislave}\\
	{\small Fakulta informatiky a informačných technológií}\\
	{\small \texttt{xcincura@stuba.sk}}
	}

\date{\small 30. september 2022}



\begin{document}

\maketitle

\begin{abstract}
\ldots
\end{abstract}



\section{Úvod}

V súčastnosti implementáciu blockchainu nájdeme len v malom počte hier, ktoré pochádzajú hlavne od malých vývojárov. Najčastejšie nájdeme využitie blockchainu v metaverse, kde zohráva dôležitú rolu pri vývoji Webu 3.0, ktorý je predovšetkým o decentralizácii. Technológia blockchainu sa používa hlavne pri predaji herných predmetov. Tento predaj sa aj tak väčšinu času realizuje na webových stránkach a nie priamo v hre. Blockchaine v hernom priemysle má omnoho väčšie využitie, ako len predaj herných predmetov.
V tradičnej hernej architektúre, používatelia nevlastnia svoje herné dáta. Ak poskytovateľ hry ukončí svoje služby, používateľ príde o všetky dáta, čiže stratia všetok čas a úsilie ktoré hre obetovali. Vďaka smart kontraktom fungujúcich na blockchaine, môžeme vytvoriť herné servery a databázy, ktoré budú hráčom poskytovať herné dáta, ktoré budú skutočne vlastniť. Keď sa používatelia rozhodnú odísť, tieto údaje je možné jednoducho preniesť do novej hry. Všeobecná forma údajov, ktoré je možné prenášať medzi hrami, sa prezentuje ako „EXPTOKEN“, čo sú mince, ktoré užívatelia zarábajú, keď trávia čas v hre, a ktoré možno vymeniť za herné predmety a iné herné výhody.

Pridanie blockchainu do terajšej hernej architektúry sme bližšie opísali v časti 2. Dôležité komponenty blockchainu sú uvedené v časti 3. Hodnotenie používateľov sme bližšie opísali v sekcii 4. V sekcii 5 sme sa venovali experimentu. Záverečná poznámky a pohľady do budúcna prináša časť 6.




\section{Herná architektúra s použitím Blockchainu} \label{nejaka}

Ako sme už spomenuli blockchain by mal v hernom priemysle obrovské využitie. Viacero veľkých spoločností prišlo s nápadom integrovania tohto prvku do svojich hier. Existuje množstvo spôsobov integrácie blockchainu. V tejto časti sa pozrieme na implementáciu blockchainu do terajšej hernej architektúry, ale taktiež na architektúru v ktorej nie sú použité tradičné prvky. Architektúra obr.~\ref{f:rozhod} by mohla byť integrovaná s tradičnými hernými servermi. Hry by sa však mohli vyvíjať aj bez spoliehania sa na tradičnú štruktúru. Pre správnu komunikáciu medzi tradičnými hernými servermi a blockchain časťou potrebujeme časť s názvom framework wrapper. Táto architektúra umožňuje herným vývojárom úplne oddeliť hernú logiku od smart kontraktov.  



\section{Komponenty blockchainu} \label{komponenty}


\subsection{Komponent hráča} \label{komponenty:hrac}


\subsection{Komponent blockchainu} \label{komponenty:blockchain}


\subsection{Komponent vývojára} \label{komponenty:vyvojar}



\section{Vyplácanie hráčov} \label{vyplacanie}



\section{Ešte dôležitejšia časť} \label{dolezitejsia}




\section{Záver a pohľady do budúcna} \label{zaver}




% týmto sa generuje zoznam literatúry z obsahu súboru literatura.bib podľa toho, na čo sa v článku odkazujete
\bibliography{literatura}
\bibliographystyle{abbrv} % prípadne alpha, abbrv alebo hociktorý iný
\end{document}
