% Metódy inžinierskej práce

\documentclass[10pt,twoside,slovak,a4paper]{article}

\usepackage[slovak]{babel}
%\usepackage[T1]{fontenc}
\usepackage[IL2]{fontenc} % lepšia sadzba písmena Ľ než v T1
\usepackage[utf8]{inputenc}
\usepackage{graphicx}
\usepackage{url} % príkaz \url na formátovanie URL
\usepackage{hyperref} % odkazy v texte budú aktívne (pri niektorých triedach dokumentov spôsobuje posun textu)

\usepackage{cite}
%\usepackage{times}

\pagestyle{headings}

\title{An Architecture for Game to Game Data Transfer Using Blockchain
\thanks{Semestrálny projekt v predmete Metódy inžinierskej práce, ak. rok 2022/23, vedenie: Igor Stupavský}} % meno a priezvisko vyučujúceho na cvičeniach

\author{Daniel Činčura\\[2pt]
	{\small Slovenská technická univerzita v Bratislave}\\
	{\small Fakulta informatiky a informačných technológií}\\
	{\small \texttt{xcincura@stuba.sk}}
	}

\date{\small 30. september 2022} % upravte



\begin{document}

\maketitle

\begin{abstract}
\ldots
\end{abstract}



\section{Úvod}

V tradičnej hernej architektúre, používatelia nevlastnia svoje herné dáta. Ak poskytovateľ hry ukončí svoje služby, používateľ príde o všetky dáta, čiže stratia všetok čas a úsilie ktoré hre obetovali. Vďaka smart kontraktom fungujúcich na blockchaine, môžeme vytvoriť herné servery a databázy, ktoré budú hráčom poskytovať herné dáta, ktoré budú skutočne vlastniť. Keď sa používatelia rozhodnú odísť, tieto údaje je možné jednoducho preniesť do novej hry. Všeobecná forma údajov, ktoré je možné prenášať medzi hrami, sa prezentuje ako „EXPTOKEN“, čo sú mince, ktoré užívatelia zarábajú, keď trávia čas v hre, a ktoré možno vymeniť za herné predmety a iné herné výhody.

Pridanie blockchainu do terajšej hernej architektúry sme bližšie opísali v časti 2. Dôležité komponenty blockchainu sú uvedené v časti 3. Hodnotenie používateľov sme bližšie opísali v sekcii 4. V sekcii 5 sme sa venovali experimentu. Záverečná poznámky a pohľady do budúcna prináša časť 6.~\ref{nejaka}.




\section{Herná architektúra s použitím Blockchainu} \label{nejaka}

Architektúra obr.~\ref{f:rozhod} by mohla byť integrovaná s tradičnými hernými servermi. Hry by sa však mohli vyvíjať aj bez spoliehania sa na tradičnú štruktúru.

\begin{figure*}[tbh]
\centering
%\includegraphics[scale=1.0]{Integrácia Blockchainu.pdf}
Aj text môže byť prezentovaný ako obrázok. Stane sa z neho označný plávajúci objekt. Po vytvorení diagramu zrušte znak \texttt{\%} pred príkazom \verb|\includegraphics| označte tento riadok ako komentár (tiež pomocou znaku \texttt{\%}).
\caption{Rozhodujúci argument.}
\label{f:rozhod}
\end{figure*}



\section{Iná časť} \label{ina}

Základným problémom je teda\ldots{} Najprv sa pozrieme na nejaké vysvetlenie (časť~\ref{ina:nejake}), a potom na ešte nejaké (časť~\ref{ina:nejake}).\footnote{Niekedy môžete potrebovať aj poznámku pod čiarou.}

Môže sa zdať, že problém vlastne nejestvuje\cite{Coplien:MPD}, ale bolo dokázané, že to tak nie je~\cite{Czarnecki:Staged, Czarnecki:Progress}. Napriek tomu, aj dnes na webe narazíme na všelijaké pochybné názory\cite{PLP-Framework}. Dôležité veci možno \emph{zdôrazniť kurzívou}.


\subsection{Nejaké vysvetlenie} \label{ina:nejake}

Niekedy treba uviesť zoznam:

\begin{itemize}
\item jedna vec
\item druhá vec
	\begin{itemize}
	\item x
	\item y
	\end{itemize}
\end{itemize}

Ten istý zoznam, len číslovaný:

\begin{enumerate}
\item jedna vec
\item druhá vec
	\begin{enumerate}
	\item x
	\item y
	\end{enumerate}
\end{enumerate}


\subsection{Ešte nejaké vysvetlenie} \label{ina:este}

\paragraph{Veľmi dôležitá poznámka.}
Niekedy je potrebné nadpisom označiť odsek. Text pokračuje hneď za nadpisom.



\section{Dôležitá časť} \label{dolezita}




\section{Ešte dôležitejšia časť} \label{dolezitejsia}




\section{Záver} \label{zaver} % prípadne iný variant názvu



%\acknowledgement{Ak niekomu chcete poďakovať\ldots}


% týmto sa generuje zoznam literatúry z obsahu súboru literatura.bib podľa toho, na čo sa v článku odkazujete
\bibliography{literatura}
\bibliographystyle{abbrv} % prípadne alpha, abbrv alebo hociktorý iný
\end{document}
